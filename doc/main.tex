\documentclass{article}
\usepackage[utf8]{inputenc}
\usepackage[T2A]{fontenc}
\usepackage[russian]{babel}
\usepackage{indentfirst}
\usepackage{amsmath}
\usepackage{euler}
\usepackage[unicode, colorlinks]{hyperref}
% \usepackage{pscyr}
\begin{document}
\author{Вова Абдрахманов}
\title{Расчёт постоянного тока в твёрдом теле методом Монте-Карло}
\maketitle
\tableofcontents
\newpage
\section{Введение}
Для изучения кинетических явлений в твёрдых телах в некоторых случаях допустимо полуклассическое приближение. Носители заряда при этом можно считать обычными частицами, спектр которых отличается от
\begin{equation}
    \varepsilon = \frac{p^2}{2m}
\end{equation}
из-за взаимодействия с кристаллической решёткой; при этом роль импульса играет квазиимпульс. При этом скорость определяется из соотношения
\begin{equation}
    \vec{v} = \frac{\partial \varepsilon}{\partial \vec{p}},
\end{equation}
а уравнение движения определяются вторым законом Ньютона.

\section{Уравнение движения}
Записывая второй закон Ньютона для носителей заряда, получаем
\begin{equation}
    \dot{\vec{p}} = \vec{F}.
\end{equation}
Так как в твердом теле носители движутся под действием электрических и магнитных полей, то в правой части стоит сила Лоренца:
\begin{equation}
    \dot{\vec{p}} = q\left(\vec{E} + \vec{v}\times\vec{B}\right).
    \label{eq:newton}
\end{equation}

Но это уравнение определяет движение в идеальной решётке. Если же в решетке есть дефекты (например, фононы), то необходимо учесть рассеяние на них. При этом между актами рассеяния движение будет описываться уравнением~\eqref{eq:newton}.

\section{Рассеяние на фононах}
Пусть плотность вероятности рассеяния на фононах из состояния \( \vec{p} \) в состояние \( \vec{p}' \) в единицу времени равна \( w(\vec{p}, \vec{p}') \). Тогда полная вероятность рассеяния в единицу времени равна 
\begin{equation}
    W(\vec{p}) = \int_{BZ} w(\vec{p}, \vec{p}') d^3 p'.
\end{equation}

Получим теперь распределение для времени между двумя последовательными процессами рассеяния. Вероятность того, что рассеяния не произойдёт за достаточно малое время \( \Delta t \) равна \( P_1 = 1 - W(\vec{p})\Delta t \). Таким образом, вероятность того, что рассеяние не произойдёт в течение промежутка времени \( t \) равна
\begin{equation}
    P(T > t) = \lim_{\max\{\Delta t_i\} \to 0} \prod_{\sum_i \Delta t_i = T} \left[1 - W(\vec{p}(t_i))\Delta t_i\right].
\end{equation}
Логарифмируя, получаем
\begin{gather}
    \ln P(T > t) = \lim_{\max\{\Delta t_i\} \to 0} \sum_{\sum_i \Delta t_i = t} \ln\left[1 - W(\vec{p}(t_i))\Delta t_i\right] = \\ = -\lim_{\max\{\Delta t_i\} \to 0} \sum_{\sum_i \Delta t_i = t} W(\vec{p}(t_i))\Delta t_i = -\int_0^t W(\vec{p}(t')) dt'.
\end{gather}
Для функции распределения отсюда получаем
\begin{equation}
    F(t) = P(T < t) = 1 - \exp\left[-\int_0^t W(\vec{p}(t')) dt'\right].
\end{equation}

Для определения времени между двумя рассеяниями при моделировании можно использовать равномерно распределённую величину \(U \in [0, 1]\). Тогда время можно отыскать из следующих соображений:
\begin{equation}
    F(t) = U,\quad \int_0^t W(\vec{p}(t')) dt' = -\ln (1 - U).
\end{equation}

Теперь было бы неплохо определить \( w(\vec{p}, \vec{p}') \). Для этого рассмотрим для начала электроны в идеальной решётке в отсутствие фононов.

\subsection{Электроны в идеальной решётке}
\subsection{Колебания решётки}
\subsection{Акустические фононы}
\subsection{Оптические фононы}
\end{document}